\section{Sparse Regularization}
\label{sec:Sparse Regularization}


\subsection{Point Cloud Consolidation}
\label{subsec:Point Cloud Consolidation}

Point cloud consolidation, known as reconstructing the geometry of a shape from scanned data, is a convenient and direct way to obtain 3D models.
It can be a preprocessing phase for some geometry problem, e.g., surface reconstruction whose result is a mesh object, with functionalities such as denoising, outlier removal, orientation, and redistribution of the input points.
However, even with high-fidelity scanners, a variety of acquisition errors, like noise, outliers, missing data(holes) or registration artifacts, are inevitable in the produced large amount of raw, dense point sets.
Then finding a robust consolidation technique has always been an active researching area.
%The following works are all $\ell_1$ norm based.

\subsubsection{$\ell_1$ median based}
\label{subsubsec:l1 median based}

\begin{figure}[ht]
  \centering
  \includegraphics[width=2.0in]{images/L1median}
  \caption{Reconstruction by projection operation. (a). noisy point-set P(green) and an arbitrary point-set Q(red) that will be projected to P to approximate P. (b),(c) are two iterative projection results. (d) is the final projection.}
  \label{fig:L1 median}
\end{figure}

Reconstruction by a projection operator, as shown in Figure~\ref{fig:L1 median}, is to approximate the origin point set(green) by iteratively projecting an arbitrary point set(red) onto itself while removing the noises or outliers.
It has an important virtue: it defines a consistent geometry based on the data points, and provides constructive means to up-sample it.

$\ell_1$ median\cite{brown1983statistical,small1990survey}, closely related to projection operator, is a statistical tool applied globally to multivariate non-parametric point-samples in the presence of noises and outliers.
Briefly, it is a robust global center of an arbitrary set of points.
Given a data set $P=\{p_{j}\}_{j\in J}\subset \mathbb{R}^3$,
the $\ell_1$ median is defined as the point $q$ obtained by minimizing the sum of Euclidean distances to the data points

\small{
\begin{equation}
 \label{eq:L1median}
 q=\arg\min_{x}\left\{ \sum_{j\in J}^{}\|p_{j}-x\| \right\}
\end{equation}
}

\paragraph{(1)}

\cite{lipman2007parameterization} applies this tool locally to constitute a parameterization-free local projection operator(LOP).
Starting with an arbitrary initial point set $X^0=\{ x{_{i}^0}  \}_{i\in I}\subset \mathbb{R}^3$(typically $|X|\ll|P|$, $|\cdot|$ is the number of point set),
LOP computes the target point positions $X$ by performing a fixed-point iteration

\small{
\begin{equation}
 \label{eq:LOP1}
 X^{k+1}=\mathop{\argmin}_{X=\{x_{i}\}_{i\in I}}\{E_1(X^{k},P)+E_2(X^{k})\},\\
\end{equation}
}
\\
where,
\small{
\begin{equation}
 \label{eq:LOP2}
 \begin{split}
 & E_1(X^{k},P)=\sum_{i\in I}^{}\sum_{j\in J}^{}\|x_{i}-p_{j}\|\theta(\|x{_i^k}-p_{j}\|),\\
 & E_2(X^{k})=\sum_{i'\in I}^{}\lambda_{i'}\sum_{i\in I\setminus\{i'\}}^{} \eta(\|x_{i}-x{_{i'}^k}\|)\theta(\|x{_i^k}-x{_{i'}^k}\|).
 \end{split}
\end{equation}
}
\\
The term $E_1$ is in fact a $localized$ version of~(\ref{eq:L1median}) by using a fast-decaying weight function $\theta(r)=e^{-r^2/(h/4)^2}$ with the finite support radius $h$,
and thus it is just $E_1$ that drives the projected points $X$ to approximate the geometry of $P$.
The term $E_2$ keeps the distribution of the points $X$ fair by incorporating local repulsion forces.

To be convenient for the following works, now we give the expression of the solution. Let $\xi{_{ij}^k}=x{_i^k}-p_{j}$ and $\delta{_{ii'}^k}=x{_i^k}-x{_{i'}^k}$, solving~(\ref{eq:LOP1}), the projection for point $x{_i^{k+1}}$ is obtained as

\small{
\begin{equation}
 \label{eq:LOP3}
 x{_i^{k+1}}=F_1(x{_i^k},P)+F_2(x{_i^k},X{_i^{'}})
\end{equation}
}
\\
where,
\small{
\begin{equation}
 \label{eq:LOP4}
 \begin{split}
 & F_1(x{_i^k},P)=\sum_{j\in J}^{}p_{j}\frac{\alpha{_{ij}^k}}{\sum_{j\in J}^{}\alpha{_{ij}^k}},\\
 & F_2(x{_i^k},X{_i^{'}})=\mu\sum_{{i'}\in I\setminus\{i\}}^{}\delta{_{ii'}^k}\frac{\beta{_{ii'}^k}}{\sum_{{i'}\in I\setminus\{i\}}^{}\beta{_{ii'}^k}},\\
 & \alpha{_{ij}^k}=\frac{\theta(\|\xi{_{ij}^k}\|)}{\|\xi{_{ij}^k}\|},
   ~\beta{_{ii'}^k}=\frac{\theta(\|\delta{_{ii'}^k}\|)|\eta'(\|\delta{_{ii'}^k}\|)|}{\|\delta{_{ii'}^k}\|}.
 \end{split}
\end{equation}
}

Intuitively, LOP distributes the points by approximating their $\ell_1$ median to achieve robustness to outliers and data noises without any local orientation information nor a local manifold assumption.
But, also because of the use of the local density parameter $h$, it may not work well when the distribution of the input points is highly non-uniform and can fail to converge.

\paragraph{(2)}

Like\cite{lipman2007parameterization}, many consolidation methods try to obtain the resulted geometry object
without estimation of normals due to the unreliability resulting from the noisy data as oppose to the fact that
oriented normals at the points play a critical role in geometry reconstruction.

To achieve a better normal estimation that requires the sampling points to be uniformly distributed,
\cite{huang2009consolidation} incorporates locally adaptive density weights into LOP, resulting in a new consolidation technique WLOP, to address the non-uniform distribution problem while taking advantage of the success of LOP in denoising and outlier removal.

They define the weighted local densities for each point $p_{j}$ in $P$ and $x_{i}$ in $X$ during the $k$th iteration by $v_{j}=1+\sum_{j'\in J\setminus\{j\}}^{}\theta(\|p_{j}-p_{j'}\|)$ and $w{_i^k}=1+\sum_{i'\in I\setminus\{i\}}^{}\theta(\|\delta{_{ii'}^k}\|)$, the term $F_1$ and $F_2$ in~(\ref{eq:LOP4}) finally becomes

\small{
\begin{equation}
 \label{eq:WLOP}
 \begin{split}
 & F_1(x{_i^k},P)=\sum_{j\in J}^{}p_{j}\frac{\alpha{_{ij}^k}/v_{j}}{\sum_{j\in J}^{}(\alpha{_{ij}^k}/v_{j})}\\
 & F_2(x{_i^k},X{_i^{'}})=\mu\sum_{{i'}\in I\setminus\{i\}}^{}\delta{_{ii'}^k}\frac{w{_{i'}^k}\beta{_{ii'}^k}}{\sum_{{i'}\in I\setminus\{i\}}^{}(w{_{i'}^k}\beta{_{ii'}^k})},
 \end{split}
\end{equation}
}
\\
The weighted local density $v$ in $F_1$ relaxes the attraction of point clusters and
repulsion force in dense areas is strengthened by the local density $w$ in $F_2$.

Here, the obtained uniformly distributed point set can largely improve the reliability of normal initialization for a second normal estimation phase.
Practically, due to the high computational effort, it may not be a preferable choice to use this consolidation technique as a preprocessing method for surface reconstruction, even though some high quality surface can be reconstructed.


\paragraph{(3)}
In LOP/WLOP, the majority of the time is spent on the evaluation of the attractive forces from all points in $P$,
so \cite{preiner2014CPF} efficiently reduce the set $P$ of unordered input points to a much more compact mixture of Gaussians $\mathcal{M}=\{w_{s},\Theta_{s}\}$ that reflects the density distribution of the points.
That is, $\mathcal{M}$ defines a probability density function(pdf) as a weighted sum of $|\mathcal{M}|$ Gaussian components

\small{
\begin{equation}
 \label{eq:CLOP1}
 f(\mathbf{x}|\mathcal{M})=\sum_{s}^{}w_{s}g(\mathbf{x}|\Theta_{s}),
\end{equation}
}
\\
where the $\Theta_{s}=(\mu_{s},\sum_{s}^{})$ are the Gaussian parameters,
$w_{s}$ are their corresponding convex weights, and
$g$ denotes the $d$-variate Gaussian pdf.
%with $g(\mathbf{x}|\mu,\sum)=|2\pi\sum_{}^{}|^{-\frac{1}{2}}e^{-\frac{1}{2}(\mathbf{x}-\mu)^{T}\sum_{}^{-1}(\mathbf{x}-\mu)}$.

They define a $continuous$ $\mathcal{F}_1$ corresponding to $F_1$ in~(\ref{eq:LOP4}) by the convex sum over the internal attraction of each single Gaussian, with convex weights $w_s$ accounting for the Gaussian's relative point mass:

\small{
\begin{equation}
 \label{eq:CLOP2}
 \mathcal{F}_1(q,\mathcal{M})=\sum_{s}^{}w_s\int_{\mathbb{R}^3}^{}
 \frac{\mathbf{x}g(\mathbf{x}|\Theta_{s})\alpha(\mathbf{x})}
 {\sum_{s'}^{w_{s'}}\int_{\mathbb{R}^3}^{}g(\mathbf{x'}|\Theta_{s'})\alpha(x')d\mathbf{x}'}
 d\mathbf{x},
\end{equation}
}
\\
and the final \textbf{closed form} is expressed as
\small{
\begin{equation}
 \label{eq:CLOP3}
 \mathcal{F}_1(q,\mathcal{M})=\frac{\sum_{s}^{}\sum_{k}^{}\int_{\mathbb{R}^3}^{}\mathbf{x}\widehat{\Omega}_{sk}(\mathbf{x})d\mathbf{x}}
 {\sum_{s}^{}\sum_{k}^{}\int_{\mathbb{R}^3}^{}\widehat{\Omega}_{sk}(\mathbf{x})d\mathbf{x}}
 =\frac{\sum_{s,k}^{}w_{sk}\mu_{sk}}
 {\sum_{s,k}^{}w_{sk}}
\end{equation}
}
\\
changing the convex sum of 3D points $p_{j}$~(\ref{eq:LOP2}) into a convex combination of the product Gaussians' means $\mu_{sk}$ with weights $w_{sk}$.
Figure~\ref{fig:L1 median consolidation} shows the results of these three methods.

This continuous method is up to 7 times faster than an optimized GPU implementation of LOP/WLOP, and achieves interactive frame rates for moderately sized point clouds though it can not automatically get the best choice of the parameters for different point set.

\begin{figure}[ht]
  \centering
  \includegraphics[width=2.5in]{images/reconstruction_L1}
  \caption{Sparse regularization: point cloud consolidation. (a): LOP\cite{lipman2007parameterization}. (b): WOLP\cite{huang2009consolidation}. (c): continuous WLOP\cite{preiner2014CPF}.}
  \label{fig:L1 median consolidation}
\end{figure}


\subsubsection{$\ell_1$ regression based}
Due to the robustness to noises and outliers of $\ell_1$ norm,
\cite{mustafa2014subdivision} develops an $\ell_1$ regression based subdivision algorithm for curve and surface fitting, 
where the size of target point cloud is largely more than that of origin data in contrast to the previous consolidation works.

For curve fitting, they try to find the best fit straight line $f(x)=\beta_1+\beta_2 x$ with observations$(x_r=r, f_r),r=-n+1,\cdots,n$.
The $\ell_1$ regression optimization is simply formulated as

\small{
\begin{equation}
 \label{eq:subdivision}
 \begin{split}
 &\beta_1, \beta_2 = \arg \min_{\beta_1,\beta_2\in\mathbb{R}}  \sum_{r=-n+1}^{n}  | f_r - (\beta_1 + \beta_2 r) |\\
 &~~~~~~~~=\arg \min_{\beta_1,\beta_2\in\mathbb{R}} F(\beta_1,\beta_2),
 \end{split}
\end{equation}
}
\\
because of the lack of differentiability, they regularize $F$ with a family of convex functional $F_{\delta}$, $\delta>0$,

\small{
\begin{equation}
 \label{eq:subdivision regularization}
 \begin{split}
 &F_{\delta}(\beta_1, \beta_2) = \sum_{r=-n+1}^{n}  h_{\delta}( f_r -\beta_1 - \beta_2 r),~\textrm{where}\\
 &h_{\delta}( f_r -\beta_1 - \beta_2 r) = [( f_r -\beta_1 - \beta_2 r)^2+\delta]^{1/2}
 \end{split}
\end{equation}
}
\\
then for a given $\delta$, the solution of (\ref{eq:subdivision}) is approximated by $\beta_{1,\delta}$ and $\beta_{2,\delta}$.
By substituting optimum $\beta_{1,\delta}$, $\beta_{2,\delta}$ into $f(x)$ and evaluating this function at 1/4 and 3/4,
the closed form of $\ell_1$ scheme for curve fitting is obtained.

With the closed form, $\ell_1$ scheme $D_{2n}$ firstly iteratively assigns weights to only $2n$ local initial points,
then gets the final fitting result(e.g.,Figure~\ref{fig:subdivision}) through subdivision rule for locations of vertices of the new mesh
and topological rule for size of added vertices and their connectivity.


\begin{figure}[ht]
  \centering
  \includegraphics[width=2.5in]{images/subdivision}
  \caption{Sparse regularization: $\ell_1$ based subdivision\cite{mustafa2014subdivision}. Parametric surface reconstructed by $\ell_1$ scheme from highly noisy parametric data with outliers.}
  \label{fig:subdivision}
\end{figure}   % denoising

\subsection{Mesh Denoising}
\label{subsec:L0 denoising}

Different from denoising in point cloud, for mesh surfaces, there are vertex connectivity and triangle quality that can be used or considered.
But how to distinguish features from noise is still a challenging problem. 

\paragraph{(1)}
In image processing, \cite{xu2011image}, aiming to smooth images, provides an algorithm for directly optimizing the $\ell_0$ norm of gradients of image colors to create piecewise constant images.
Let $\mathbf{c}$ be a vector of pixel colors and $\triangledown \mathbf{c}$ be a vector of gradients of these colors.
They formulates the smooth problem as

\small{
\begin{equation}
 \label{eq:imagesmooth}
 \min_{\mathbf{c}}|\mathbf{c}-\mathbf{c}^{*}|^2+|\triangledown \mathbf{c}|_0
\end{equation}
}
\\
where $\mathbf{c}^{*}$ represents the original image colors to provide a data fidelity term.

A natural extension to triangulated meshes is to design a discrete differential operator to replace $\triangledown c$ that is zero when the surface is flat for arbitrary triangulations irrespective of the rotation or translation of the surface.
This constraint implies that some form of second order information rather than the first order information provided by $\triangledown c$ is needed, e.g.,  the discrete Laplacian operator\cite{pinkall1993computing} which is computed as a weighted combination of a vertex and its one-ring where the weights are given by cotangents of angles of the triangles.
However,
\parpic[r]{\label{fig:failureLaplaciandenoise}\includegraphics[width=0.4\linewidth]{images/denoise12}}
the vertex-based Laplacian only constrains the mean curvature vector as opposed to a metric that should directly measure sharpness per edge, then the optimization fails to reproduce sharp features well shown as the right figure.

\parpic[r]{\label{fig:edgeoperator}\includegraphics[width=0.5\linewidth]{images/denoise11}}
\cite{he2013mesh} generalizes this construction of the vertex-based cotan operator to an operator that acts directly on an edge

\small{
\begin{equation}
 \label{eq:edgecotanoperator}
 D(e) := {\left[ \begin{array}{c}
 \frac{\vartriangle_{1,2,3}((p_4-p_3)\cdot(p_3-p_1))}{\vartriangle_{1,3,4}((p_1-p_3)\cdot(p_3-p_2))} \\
 \frac{\vartriangle_{1,3,4}}{\vartriangle_{1,2,3}+\vartriangle_{1,3,4}} \\
 \frac{\vartriangle_{1,2,3}((p_3-p_1)\cdot(p_1-p_4))}{\vartriangle_{1,3,4}((p_2-p_1)\cdot(p_1-p_3))} \\
 \frac{\vartriangle_{1,2,3}}{\vartriangle_{1,2,3}+\vartriangle_{1,3,4}}
 \end{array}
 \right]}^{T}
 \left[ \begin{array}{c}
 p_1 \\ p_2 \\ p_3 \\ p_4
 \end{array}
 \right]
\end{equation}
}

Then the extended optimization problem is to make the edge operator sparse formulated as following

\small{
\begin{equation}
 \label{eq:L0 denoise}
 \min_{p,\delta}|p-p^{*}|^2+\alpha|R(p)|^2+\beta|D(p)-\delta|^2+\lambda|\delta|_0
\end{equation}
}
\\
where $p$ are the vertices of the shape, $p^{*}$ are their initial positions, $D(p)$ is a vector where the $i^{th}$ entry corresponds to the area-based edge operator applied to the $i^{th}$ edge, and $R(p)$ is a regularization term. Figure~\ref{fig:L0 denoise}gives one denoised result with sharp features.

\begin{figure}[ht]
  \centering
  \includegraphics[width=2.5in]{images/denoise1}
  \caption{Sparse regularization: mesh denoising\cite{he2013mesh}. Left: initial surface. Center: surface corrupted by Gaussian noise in random directions with standard deviation $\sigma=0.4l_{e}$($l_{e}$ is the mean edge length). Right: denoising result. The wireframe shows folded triangles as red edges.}
  \label{fig:L0 denoise}
\end{figure}

\begin{figure*}[ht]
  \centering
  \includegraphics[width=5in]{images/denoise2_1}
  \caption{Sparse regularization: mesh denoising\cite{wang2014decoupling}. (a) is the two-dimensional illustration for their key observation. (b) is a denoising example.}
  \label{fig:decoupling}
\end{figure*}

\paragraph{(2)}
Like most previous methods, how to tune the parameters shown in~(\ref{eq:L0 denoise}) has not theoretical guarantee and the computation of differential properties for distinguishing noise from feature is unreliable and unstable.

To address these problems, \cite{wang2014decoupling} presents a two-phase approach for decoupling features and noise on discrete surfaces.
Figure~\ref{fig:decoupling}(a) gives a two-dimensional curve as the illustration for their key observation: any surface is piecewise $C^2$, that is, a surface consists of two parts: $C^2$ smooth part and $C^0$ feature part which can be transformed into a sparse signal by applying the Laplacian operator.
As such, the denoising problem is divided into two phase: smooth part(base mesh) estimation and recovering features from the corrupted feature part.

They firstly get a base mesh by denoising the input data using a global Laplacian regularization smoothing optimization, in which the smoothness parameter is automatically chosen by adopting the generalized cross-validation scheme,
then decouple the features $x$ and noises simultaneously from the noisy feature part $y$ via the $\ell_1$ analysis compressed sensing optimization

\small{
\begin{equation}
 \label{eq:decoupling}
 \min_{x}\|Lx\|_1~~ s.t. ~ \|y-x\|_2 \le \epsilon
\end{equation}
}

Finally, combining the denoised feature part and the obtained base mesh reduces the final denoising result. Note that it is the first time noise and features are analyzed and separated in such an elegant manner with guarantees by statistical theory which is much exciting and sightworthy in the smoothing optimization.   % reconstruction based on L_1 median
\subsection{Shape Matching}
\label{subsec:Shape Matching}


Here, we give shape matching a more extensive definition: finding the correspondence(point-wise, pair-wise) between two rigid or non-rigid deformable geometric data sets.

\subsubsection{Rigid registration}
\label{subsubsec:Rigid registration}

Rigid registration aims at finding a suitable set of corresponding points on source and target point set.
The $Iterative~Closest~Point$(ICP) addresses this problem by assuming the input data to be in coarse alignment.
Under this assumption, a set of correspondences can be obtained by querying closest points on the target geometry.
Given two surfaces $\mathcal{X}$, $\mathcal{Y}$, it is formulated as

\small{
\begin{equation}
 \label{eq:ICP}
 \mathop{\argmin}_{R,\mathbf{t}}\int_{\mathcal{X}}^{}\varphi(R\mathbf{x}+t,\mathcal{Y})d\mathbf{x}+I_{\mathcal{SO}(k)}(R)
\end{equation}
}
\\
where $R$ is a rotation matrix,
$\mathbf{t}$ is a translation vector,
$\mathbf{x}$ is a point on the source geometry.
The quality of a registration is evaluated by the metric
$\varphi(\mathbf{x},\mathbf{y})=\|\mathbf{x}-\mathbf{y}\|{_2^2}$, i.e.,
classical ICP is in a least-square sense which would fail with outliers.

Now that sparse regularization methods excels in processing data set with noises or outliers,
\cite{bouaziz2013sparse} tries to formulate the local alignment problem as recovering rigid transformation that minimizes the number of zero distances between two correspondences.
They adopt $\ell_{p}$($0\le p\le1$) norm based sparse regularizer to obtain an heuristic-free, robust rigid registration algorithm by modifying

\small{
\begin{equation}
 \label{eq:permutedsparse}
 \varphi(\mathbf{x},\mathbf{y})=\|\mathbf{x}-\mathbf{y}\|{_2^{p}}
\end{equation}
}
\\
About $\ell_{p}$ norm, \cite{chartrand2007exact} shows that $\ell_{p}$ norms with $p<1$ outperform the $l_1$ norm in inducing sparsity and \cite{elad2010sparse} also illustrates the tendency of $\ell_{p}$($0<p<1$)norms to drive results to become sparse.
Figure~\ref{fig:sparseICP} is the registration results of sparse ICP under different values of $p$ among which it can be found that $0<p<1$ reduces better results,
but the value of $p$ is selected according to the experiments to offer a trade-off between performance and robustness which may make the sparse ICP unpractical.

\begin{figure}[ht]
  \centering
  \includegraphics[width=3in]{images/sparseICP}
  \caption{Sparse regularization: rigid registration results using sparse ICP\cite{chartrand2007exact} under different $l_{p}$ norms.}
  \label{fig:sparseICP}
\end{figure}



\subsubsection{Non-rigid shape matching}
\label{subsubsec:non-rigid shape matching}

\paragraph{(1)Local functional basis}
For a 3D surface, the invariance of intrinsic properties to extrinsic transformations should always be handled.
The eigenfunctions of the Laplace-Beltrami operator just define this kind of basis, manifold harmonic basis(MHB), 
which is unique and characteristic of the geometric and topological properties of the shape.
Now we first have a look at one work about it, which is closely related to the following non-rigid shape matching algorithm.


The Laplace-Beltrami operator $\triangle$ on a 2D manifold surface embedded in 3D space induces the eigenfunctions $\{\phi_{k}\}$ satisfying the equations

\small{
\begin{equation}
 \label{eq:MHB}
 -\triangle\phi_{k}=\lambda_{k}\phi_{k},~k\in\mathbb{N},\lambda_{k}\in\mathbb{R},
\end{equation}
}
\\
where $\lambda_{k}$ are the eigenvalues of the operator.
With their global spatial support, MHB have been used for many applications.

But as we have known that many times only locality can reduce a good result, like deformation(\cite{zhang2014local} mentioned above), correspondence.
So, to produce an intrinsic shape basis with local spatial support while taking advantage of MHB simultaneouly, \cite{neumann2014compressed} proposes the $compressed~manifold~basis(CMB)$, whose individual basis functions are called $compressed~manifold~modes(CMMs)$, by adding a sparsity inducing $\ell_1$ norm into~(\ref{eq:MHB})


\small{
\begin{equation}
 \label{eq:CMBA}
 \begin{split}
 & \min_{\phi_{k}}  \sum_{k=1}^{K} \langle \phi_{k},\triangle\phi_{k} \rangle  +  \mu | \phi_{k} |_1,\\
 & ~\textrm{s.t.}~ \langle \phi_{k}, \phi_{j}  \rangle= \delta_{kj},
 \end{split}
\end{equation}
}
\\
where $\delta_{kj}$ is the Kronecker delta used to enforce orthogonality of the eigenfunctions and $\mu$ is used to control the sparsity.
For a triangle mesh with $N$ vertices, discretizing the Laplacian $\triangle$ using a sparse matrix $L\in\mathbb{R}^{N\times N}$ with cotangent weights in previous work, and incorporating a lumped mass matrix $M$, containing the vertex areas along its diagonal making the eigenbasis independent of the mesh resolution, the discretization of~(\ref{eq:CMBA}) becomes

\small{
\begin{equation}
 \label{eq:discreteCMBA}
 \begin{split}
 & \min_{\Phi} \textrm{Tr}(\Phi^{T}L \Phi)+\mu\| \Phi \|_1,\\
 & ~\textrm{s.t.}~ \Phi^{T}M\Phi=I.
 \end{split}
\end{equation}
}
\\
here, $\Phi\in\mathbb{R}^{N\times K}$ contains the first $K$ eigenvectors corresponding to the matrix columns.
Solving~(\ref{eq:discreteCMBA}), the obtained orthogonal $compressed~manifold~modes(CMMs)$ could automatically identify key shape features of the underlying mesh, as shown in Figure~\ref{fig:CMB}. As such, it can be used for shape matching which involves robust feature detection.


\begin{figure}[ht]
  \centering
  \includegraphics[width=2.5in]{images/CMB}
  \caption{Sparse regularization: local functional basis\cite{zhang2013point}. The proposed compressed manifold modes(CMMs) have local support and are confined to specific local features like protrusions and ridges. 8 of the CMMs were found for the 8 protrusion at the corner(2 shown here), 6 concentrate at each of the dents(2 shown here), and 12 CMMs automatically form at the valleys between the protrusions.}
  \label{fig:CMB}
\end{figure}



\paragraph{(2)Non-rigid shape matching}
Matching of deformable shapes is a notoriously difficult problem which results in the number of degrees of freedom growing exponentially with the number of matched points.

Recently, \cite{ovsjanikov2012functional} introduces a functional representation for correspondences 
which are modeled as the correspondences between functions on two shapes rather than points.
Mathematically, let $X$ and $Y$ be two shapes equipped with bases $\{\phi_{i}\}_{i\ge1}$ and $\{\psi_{j}\}_{j\ge1}$ respectively,
any real function $f: X\to \mathbb{R}$ and $g=T(f): Y\to \mathbb{R}$ can be represented as $f=\sum_{i\ge1}^{}a_{i}\phi_{i}$ and $g=\sum_{j\ge1}^{}b_{j}\psi_{j}$.
Taking discretized functions $\phi_{i}$ and $\psi_{j}$ as the columns of bases matrices $\Phi$ and $\Psi$,
the function vectors can be represented as
$\mathbf{f}=\Phi \mathbf{a}$ and
$\mathbf{g}=\Psi \mathbf{b}$, and then from
$\Psi \mathbf{b}=T(\Phi \mathbf{a})=\Psi C^{T}\mathbf{a}$,
the relationship between two coefficients is clear that $\mathbf{b^{T}}=\mathbf{a^{T}}C$.
Thus, the matrix $C$ fully encodes the linear map $T$ between the functional spaces.


In case the shapes $X$ and $Y$ are isometric and the corresponding Laplace-Beltrami operators have simple spectra mentioned above,
the harmonic bases(Laplacian eigenfunctions) have a compatible behavior, $\psi_{i}=T(\phi_{i})$ such that $c_{ij}=\delta_{ij}$.
Choosing the discretized eigenfunctions of the Laplace-Beltrami operator as $\Phi$ and $\Psi$ causes every low-distortion correspondence being represented by a nearly diagonal, and therefore very sparse matrix $C$.

Based on the above theory, \cite{pokrass2013sparse} firstly gets two collections of similar functions $\{f_{i}:X\to \mathbb{R}\}$ and $\{g_{j}:Y\to \mathbb{R}\}$ using some region detection process like\cite{litman2011diffusion}. \parpic[r]{\label{fig:regionmatching}\includegraphics[width=0.3\linewidth]{images/matching_function}}
As shown in the right figure, different colors represent different functions and the correspondence of these two collections of functions is unknown, i.e., we do not know to which $g_{j}$ in $Y$ a $f_{i}$ in $X$ corresponds.
\cite{pokrass2013sparse} adopts an unknown permutation matrix $\Pi$ to express this ordering. Finally, the robust permuted sparse coding is formulated as following

\small{
\begin{equation}
 \label{eq:non-rigid shape matching}
 \min_{C,O,\Pi}\frac{1}{2} \|\Pi B - AC - O \|{_{F}^2} + \lambda\| W \odot C\|_1+\mu\|O\|_{2,1}
\end{equation}
}
\\
where $W$ is assigned with larger weights in off-diagonal part and small weights in diagonal part to promote diagonal solutions, $\|O\|_{2,1}$ promotes row-wise sparsity allowing to absorb the errors in the data term corresponding to the rows of $A$ having no corresponding rows in $B$.
From the formulation we know that this method relies on the region detection technique and assumption: near-isometric shapes.
Figure~\ref{fig:non-rigid matching} shows the correspondences between non-isometric shapes.

\begin{figure}[ht]
  \centering
  \includegraphics[width=3in]{images/matching_L1}
  \caption{Sparse regularization: non-rigid shape matching \cite{wang2014decoupling}. First row: point-to-point correspondences between different non-isometric shapes. Second row: point-to-point correspondence between SHREC shapes undergoing nearly isometric deformations and noise.}
  \label{fig:non-rigid matching}
\end{figure}


\subsubsection{Co-segmentation}
\label{subsubsec:co-segmentation}

Co-segmentation aims to consistently segment a group of shapes and obtain the correspondence between resulted segments simultaneously,
as the right figure in Figure~\ref{fig:co-segmentation} shows, corresponding parts are labeled in the same colors.
To be more intuitive and efficient, \cite{hu2012co} processes co-segmentation on patch-level instead of face-level.

Thus they firstly over-segment all the models(left in Figure~\ref{fig:co-segmentation}) followed by calculating their feature vectors using some feature descriptors(in this paper, they adopt $H=5$ feature descriptors).
For example, Figure~\ref{fig:co-segmentationAGD} shows the colormaps of average geodesic distance(AGD) features of two tables with over-segmented patches.
They define the feature vector as a histogram of the feature measurement on the triangles of that patch.
It is obvious that two corresponding patches have similar distributions, that is, their feature vectors lie in a common subspace generated by standard basis corresponding to these nonzero entries.
Based on this observation, they regard co-segmentation as a subspace clustering problem since the final segments are all clustering of over-segmented patches.

\begin{figure}[ht]
  \centering
  \includegraphics[width=2.5in]{images/co-segmentationAGD}
  \caption{Sparse regularization: co-segmentation\cite{hu2012co}. Colormaps of AGD features of two tables with over-segmented patches. The AGD feature vectors of the two patches(marked in rectangles) from each table's leg have similar distribution, as shown in histograms in the middle. It can be seen that these two feature vectors lie in the common subspace generated by standard basis corresponding to the nonzero entries.}
  \label{fig:co-segmentationAGD}
\end{figure}


Since each data point(here is the feature vector) in a union of linear subspaces can always be represented as a linear combination of the points belonging to the same linear subspace,
the combination will be sparse if the point is written as a linear combination of all other points.
Following\cite{elhamifar2009sparse,wang2011efficient}, finding the sparse combination matrix for the single-feature co-segmentation is formulated as

\small{
\begin{equation}
 \label{eq:SSC}
 \begin{split}
 &\min_{W_{h}}\|X_{h}W_{h}-X_{h}\|{_{F}^2}+\lambda\|W_{h}^{T}W_{h}\|_{1,1} \\
 &\mathrm{s.t.}~W_{h}\ge0,~\textrm{diag}(W_{h})=0
 \end{split}
\end{equation}
}
\\
where $h$ corresponds to the $h$-th feature descriptor,
the feature matrix $X_{h}=[x_{h1},x_{h2},\cdots,x_{hN}]$ is constructed with the feature vector $x_{hi}$ of the $i$-th patch($i=1,2,\cdots,N$).
$\|W_{h}^{T}W_{h}\|_{1,1}$, as a penalty item, favors the sparsity of the optimal solution $\overline{W_{h}}$ of which each entry measures the linear correlation between two points in the meshes.
After defining the affinity matrix $S=(s_{ij})$ as $s_{ij}=|\overline{w_{h}}_{ij}|+|\overline{w_{h}}_{ji}|$, the NCut method\cite{shi2000normalized} is applied to get the co-segmentation results.

In fact, single one feature is not enough for co-segmenting different categories of models.
To find the most similar patch pairs considering all selected features some of which the corresponding patches may not be similar ,
\cite{hu2012co} adds the consistent multi-feature penalty to ensure the co-segmentation results consistent with different feature spaces by combing $H$ feature descriptors

\small{
\begin{equation}
 \label{eq:coseg1}
 \begin{split}
 &\min_{W_{1},\cdots,W_{H}}\sum_{h=1}^{H}\mathcal{F}(W_{h})+\mathcal{P}_{cons}(W_1,W_2,\cdots,W_H)\\
 &\mathrm{s.t.}~W_{h}\ge0,~\textrm{diag}(W_{h})=0,h=1,2,\cdots,H.
 \end{split}
\end{equation}
}
\\
where $\mathcal{P}_{cons}$ is the penalty on the matrices $W_1,W_2,\cdots,W_H$

\small{
\begin{equation}
 \label{eq:coseg2}
 \mathcal{P}_{cons}(W_1,W_2,\cdots,W_H)=\alpha\|W\|_{2,1}+\beta\|W\|_{1,1}\\
\end{equation}
}
\\
here the $H\times N^2$ matrix $W$ is formed by concatenating $W_1,W_2,\cdots,W_H$(each matrix in one row) together:

\small{
\begin{equation}
 \label{eq:coseg3}
 W = {\left[ \begin{array}{cccc}
 (W_1)_{11} & (W_1)_{12} & \cdots & (W_1)_{N^2}\\
 (W_2)_{11} & (W_2)_{12} & \cdots & (W_2)_{N^2}\\
 \vdots & \vdots & \ddots & \vdots\\
 (W_{H})_{11} & (W_{H})_{12} & \cdots & (W_{H})_{N^2}
 \end{array}
 \right]}
\end{equation}
}
\\
the $\ell_{2,1}$ penalty induces column sparsity of $W$ such that most columns of $W$ are shrunken to be entirely zero, which means that the corresponding pairs of patches will likely not be in the same cluster.
The $\ell_{1,1}$ penalty induces the sparsity within each column, then for each similar patch pair, only a subset of features are actually used to measure their similarity.
Combining these two penalties enables the prominent features to pop up and guarantees the sparsity consistency of the matrices $W_1,W_2,\cdots,W_H$.

Notice that without $\mathcal{P}_{cons}$, the formulation~(\ref{eq:coseg1}) will reduce to a naive solution which is exactly the same as applying subspace clustering to each feature matrix $X_{h}$ independently.

\begin{figure}[ht]
  \centering
  \includegraphics[width=3in]{images/co-segmentation}
  \caption{Sparse regularization: co-segmentation\cite{hu2012co}. Left shows the over segmented patches that will be clustered to get the co-segmentation result.}
  \label{fig:co-segmentation}
\end{figure}

  % shape matching



\subsection{Skeleton Extraction}
In section~\ref{subsubsec:l1 median based}, we have introduced much information about $\ell_1$ median and its success in point cloud consolidation.
Except for reducing 2D surface that approximate origin point set,
\cite{huang2013l1} observed that adapting $\ell_1$ medians $locally$ to a point set representing a geometric shape also gives rise to a $one~dimensional$ structure which can be seen as a localized center of the shape, i.e., a medial curve skeleton, which can be used for shape abstraction and consequently an effective tool for shape analysis and manipulation\cite{cornea2007curve}.

Without building any point connectivity or estimating point normals,
by modifying the repulsion term $E_2$ in~(\ref{eq:LOP2}) and proposing a different weighted density parameter that can also be named WLOP\cite{huang2009consolidation},
they project point samples onto their local centers with growing neighborhood and push the projected samples via conditional regularization to obtain a uniform distribution of samples along skeleton branches. To deal with some data errors like holes, they also do more processing which is out of our scope.
Figure~\ref{fig:skeleton extraction} shows an example.

\begin{figure}[ht]
  \centering
  \includegraphics[width=2.5in]{images/skeleton_L1}
  \caption{Sparse regularization: skeleton extraction\cite{huang2013l1}. Given an unorganized, unoriented, and incomplete raw scan with noise and outliers(b), a complete and quality curve skeleton is extracted(c).}
  \label{fig:skeleton extraction}
\end{figure}


\subsection{Deformation-Constrained Modeling}

Constrained modeling is an important tool for the modification of 3D geometric models.
Local control, in contrast to some global algorithms, is designed for adjusting $as~ few~ vertices~ as~ possible$ in order not to influence the regions that are already satisfactory.
To automatically explore a local deformation which satisfies all constraints, \cite{deng2013exploring} gives a novel framework using $\ell_{2,1}$ sparse regularization penalty.

For a mesh with vertices $\{v_{i}\}{_{1}^{n}}$,
$\mathbf{p}=[\mathbf{p}{_{1}^{T}},\cdots, \mathbf{p}{_{n}^{T}}]^{T}\in \mathbb{R}^{3n}$ is the position vector.
$E_{j}(\mathbf{p})=0(j=1,\cdots,m)$ denotes the constraints satisfactory.
Then from a given mesh $\mathbf{p}^0$ and the target positions of vertices(handles) $\{v_{i}|i\in\Gamma\}$ specified by a user,
the deformation is achieved by computing the displacement
$\mathbf{d}=[\mathbf{d}{_{1}^{T}},\cdots, \mathbf{d}{_{n}^{T}}]^{T}\in \mathbb{R}^{3n}$, with
$\mathbf{d}_{k}\in\mathbb{R}^3$ corresponding to vertex $v_{k}$,
by the following optimization problem

\small{
\begin{equation}
 \label{eq:localmodeling1}
 \begin{split}
 & \min_{\mathbf{d}}  \frac{\omega_{h}}{2} \sum_{i\in\Gamma}^{} \| \mathbf{d}_{i}- \widetilde{\mathbf{d}_{i}} \|{_2^2}
                      +\frac{\omega_{s}}{2} \sum_{i\notin\Gamma}^{} \| \mathbf{d}_{i} \|_2
                      +\frac{\omega_{f}}{2} \| \mathbf{Ed} \|{_2^2},\\
 & ~\textrm{s.t.}~ E_{j}(\mathbf{p}^0+\mathbf{d})=0,~j=1,\cdots,m,
 \end{split}
\end{equation}
}
\\
where $\widetilde{\mathbf{d}_{i}}$ in the first term is the target displacement of the handle vertex $v_{i}$,
the second $\ell_{2,1}$ term minimizing the $\ell_1$ norm of vector $[\|\mathbf{d}_1\|_2,\cdots,\|\mathbf{d}_n\|_2]$ reduces the sparsity of $\mathbf{d}$,
and last term is for a smooth displacement for a nice shape.
With fixed weight $\mathbf{\omega}=(\omega_{h},\omega_{s},\omega_{f})$, the resulted single solution $\mathbf{d}$ may not satisfy the user's intent.

Then to enrich the solution while preserving nice shapes of meshes,
based on the modified mesh $\mathbf{p}^0+\mathbf{d}$ denoted as $\mathbf{p}^{+}$,
they compute a local modification space
$S_{\mathbf{p}^{+}}=\{ \mathbf{p}^{+}+\mathbf{t}~|~\mathbf{t}\in \mathcal{S}_{\mathbf{p}^{+}} \}$ spanned by an orthonormal basis $\mathbf{t}_1,\cdots,\mathbf{s}_1$,
where $\mathcal{S}_{\mathbf{p}^{+}}$ is a linear subspace of
$\mathcal{T}_{\mathbf{p}^{+}}=\{ \mathbf{t}~|~\nabla E_{j}(\mathbf{p})\cdot \mathbf{t}=0,j=1,\cdots,m \}$
representing displacements from $\mathbf{p}$ that satisfy the constraints up to first order.
Also taking shape quality and sparsity into consideration, this problem is finally formulated as

\small{
\begin{equation}
 \label{eq:localmodeling1}
 \begin{split}
 & \min_{\mathbf{t}}  \frac{\beta_{f}}{2} \| \mathbf{Ed} \|{_2^2}
                      +\frac{\beta_{h}}{2} \sum_{i=1}^{s} \| \mathbf{t}_{i} \|{_2^2}
                      +\frac{\beta_{s}}{2} \| \mathbf{t} \|_{2,1}
                      -\frac{\beta_{c}}{2} C(\mathbf{t}),\\
 & ~\textrm{s.t.}~ \| \mathbf{t} \|_2=1,~\mathbf{Jt}=\mathbf{0},~\mathbf{B}^{T}\mathbf{t}=0.
 \end{split}
\end{equation}
}
\\
obviously, the third term is for the sparse displacement $\mathbf{t}$
with the first two term for shape quality and
the last term for the sparsity of $\mathbf{d}+\mathbf{t}$.


After the interactive exploration, the final result is optimized to fully satisfy the set of constraints(Figure~\ref{fig:localmodeling}).

\begin{figure}[ht]
  \centering
  \includegraphics[width=3.0in]{images/localmodeling}
  \caption{Sparse regularization: constraint modeling\cite{deng2013exploring}. Local modifications of a constrained mesh. A glass structure composed of planar quads is locally deformed by exploring a subspace encoding local planar modifications of its central zone.}
  \label{fig:localmodeling}
\end{figure}


\subsection{Total Variation(TV) Based Applications}
\label{subsec:TV Applications}

%Total variation(TV) For a function$f(x)$ defined on a domain $\Omega\subseteq\mathbb{R}^n$, its TV is defined as
%
%\small{
%\begin{equation}
% \label{eq:continuousTV1}
% J_{f}=\int_{\Omega}^{}| \nabla f |dx,
%\end{equation}
%}
%\\
%where $| \nabla f |$ is the $\ell_2$-norm of the gradient $\nabla f$, i.e.,
%
%\small{
%\begin{equation}
% \label{eq:continuousTV2}
% | \nabla f |= \sqrt{ \sum_{i=1}^{} ( \frac{\partial f} {\partial x_{i}} )^2}.
%\end{equation}
%}
%\\

Total variation(TV) has been a popular tool for image processing tasks,
such as denoising, reconstruction, and segmentation\cite{chambolle2010introduction},
the underlying model for TV methods aims at exploiting the sparsity of the gradient of the image.
The discrete variant yields the following convex objective function

\small{
\begin{equation}
 \label{eq:descreteTV}
 TV(u)=\sum_{i,j}^{}\|\mathcal{D}_{i,j}u\|_2
\end{equation}
}
\\
where $\mathcal{D}_{i,j}$ is the discrete gradient operator at pixel $(i,j)$ and $u$ is a vector containing the gray-level pixel values. TV methods filter the image by minimizing $TV(u)$ which is in fact the $\ell_1$ norm of the vector $[\cdots~\|\mathcal{D}_{i,j}u\|_2~\cdots]$.

Since TV is designed for images, it is not directly applicable to geometry processing problem.
As we have stated, the key point is to find some form of second order information.


\subsubsection{Point cloud consolidation}
\label{subsubsec:TVPoint cloud consolidation}

In section~\ref{subsec:Point Cloud Consolidation}, there have been several consolidation works based on $\ell_1$ median.
Here we introduce one more well-known work.
Similar to the sparse gradient minimization, and based on the observation that the gradients(normal differences) of smooth surface normals(normal differences) are sparse,
\cite{avron2010L1} formulates the piecewise smoothness reconstruction problem as a sparse minimization of orientation differences and position projections as following

\small{
\begin{equation}
 \label{eq:TVconsolidation1}
 \begin{aligned}
 &N^{out} =\mathop{\argmin}_{N} \sum_{(p_{i},p_{j})\in E}^{} w_{i,j}\|n_{i}-n_{j}\|_2\\
 &\mathrm{s.t.}~\forall i~\|n_{i}-{n_{i}}^{in}\|_2\le \gamma_{n}
 \end{aligned}
\end{equation}
}
\small{
\begin{equation}
 \label{eq:TVconsolidation2}
 X^{out} = \mathop{\argmin}_{X} \sum_{(p_{i},p_{j})\in E}^{} w_{i,j}|n_{i,j}\cdot (x_{i}-x_{j})|
\end{equation}
}
\\
where $\{n_{i}\}$ denote the surface normals, $\{x_{i}\}$ denote the point positions and $\{w_{i,j}\}$ is a set of the weight whose role is to achieve lower-than-$\ell_1$ sparsity.

Convexity of these two problems allows for finding a global optimum and deriving efficient solvers.
Figure~\ref{fig:TV consolidation} shows a well reconstructed example with sharp features.
Due to the global nature, this algorithm is extremely slow.
And it may fail for the point set with severe noises and outliers.

\begin{figure}[ht]
  \centering
  \includegraphics[width=3in]{images/TV_consolidation}
  \caption{Sparse regularization: TV based point cloud consolidation\cite{lipman2007parameterization}. The Armadillo statue(left) is scanned generating a noisy point-cloud(middle). The right figure shows the consolidation result preserving the sharp features.}
  \label{fig:TV consolidation}
\end{figure}


\subsubsection{Decomposition}
\label{subsubsection:Decompsition}

Mesh decomposition aims at segmenting a mesh into meaningful parts which are consistent with user intention, geometric mesh attributes, and human shape perception.
Generally, the elements within the same segment should have high similarity, the segment boundary should be tight and smooth as well as matching human perception, and obviously the segmentation should reflect significant features.

Motivated by the preceding observation, \cite{zhang2012variational} proposes a new method based on the Mumford-Shah model(M-S model)\cite{mumford1989optimal} that has proven successful in image segmentation, i.e., this method is also an extension from 2D images to 3D meshes.

In 2D image $I:\Omega\rightarrow \mathbb{R}^2$, the Mumford-Shah image segmentation is to find a partition $\Omega=\bigcup{_{i=1}^{k}}\Omega_{i}$, where $\Omega_{i}$ are pairwise disjoint, and numbers $c_{i}$ for $\Omega_{i}$ formulated as

\small{
\begin{equation}
 \label{eq:TV image M-S}
 \inf_{\Omega_{i},c_{i}} \sum_{}^{k}
 ( \int_{\Omega_{i}}^{} (I(x)-c_{i})^2dx + \frac{\mu}{2} | \partial\Omega_{i} | ),
\end{equation}
}
\\
where $\mu$ is a constant, $\partial \Omega$ and $|\partial \Omega|$ represent the boundary and the boundary length of segment $\Omega$, respectively.
The first data term measures the consistency of each segment and the second regularization term measures the boundary length.

To segment a 3D triangulated surface $M$, \cite{zhang2012variational} convexifies this difficult nonconvex problem~(\ref{eq:TV image M-S}) based on TV to get a new version of M-S model

\small{
\begin{equation}
 \label{eq:TV surface M-S}
 \min_{\mathbf{u}\in K, \chi_{i}} \left \{
 \int_{M}\langle\mathbf{u}(x), \mathbf{s}(x)\rangle +
 \mu g(x)| \nabla_{M}\mathbf{u}(x) | d\sigma
 \right\},
\end{equation}
}
\\
where $K$ is the set of vector functions $\mathbf{u}=(u_1,\cdots,u_{k})^{T}:M\rightarrow \mathbb{R}^{k}$ satisfying that for all $x\in M$ and $i\in [1,\cdots,k],u_{i}(x)\geq0$ and
$\sum_{i=1}^{k}u_{i}(x)=1; \mathbf{s}(x)=(s_1(x),\cdots,s_k(x))^T$ is a $k$-dimensional vector with $s_i(x)=(\mathbf{f}(x)-\chi_{i})^{T}(\mathbf{f}(x)-\chi_{i})$ indicating
the affinity vector $\chi_{i}$ that is associated with $M_{i}$ which is a segment.
The $\mathbf{f}(x)$ is a predefined multichannel function, which is constructed with the eigenvectors of the Laplacian matrix of the dual graph of $M$, for representing some attributions of $x$ over mesh $M$ similar to the RGB function for an image.

In~(\ref{eq:TV surface M-S}), if the affinity of $x$ with segment $M_{i}$ is large, $u_{i}(x)$ will tend to be small in order to reach the minimization, thus $u_{i}(x)$ can be viewed as the probability of $x$ being assigned to segment $M_{i}$ and $\mathbf{u}(x)$ can be used as a classification function for the segmentation.

Now that the regularization term in~(\ref{eq:TV surface M-S}) is to constrain the boundary with some geometric difference information between segments, it may fail for the relative smooth models.



\begin{figure}[ht]
  \centering
  \includegraphics[width=3in]{images/segmentation}
  \caption{Sparse regularization: TV based mesh decomposition\cite{zhang2012variational}. Decomposition results where the models are taken from the Princeton Segmentation Benchmark\cite{chen2009benchmark}. One mesh is shown for each category. The segmentation results match results match human perception well in not only the cutting boundaries but also the number of segments.}
  \label{fig:segmentation}
\end{figure}


\subsubsection{Barycentric coordinates}
\label{subsubsec:Barycentric coordinates}

Barycentric~coordinates provide a simple and convenient way of interpolating values from a set of control points over the interior of a domain, using weighted combinations of values associated with different control points.

Many barycentric coordinates typically are of global nature, meaning that the interpolated value depends on many, potentially $all$, control points. Besides the lack of locality and scalability, the interpolation is computationally expensive since it involves a weighted sum of all control points for each interior vertex.
Thus, barycentric coordinates with locality provide benefit in terms of storage requirements as well as computational cost.

\cite{zhang2014local} introduces a novel method to derive $local~barycentric~coordinates$(LBC), which depend only on a small number of control points.
Given a set of control points $\mathbf{c}_1, \cdots, \mathbf{c}_n$ in $\mathbb{R}^2$ or  $\mathbb{R}^3$ which are the vertices of a closed control cage, and let the domain bounded by the cage.
They want to find a function $w_{i}$: $\Omega\rightarrow\mathbb{R}$ for each $\mathbf{c}_{i}$, such that $[w_1(\mathbf{x}), \cdots, w_n(\mathbf{x})]$ is a set of generalized barycentric coordinates of $\mathbf{x}\in\Omega$ with respect to the control points $\{\mathbf{c}_{i}\}$ and is used for interpolating function values $f(\mathbf{c}_1), \cdots, f(\mathbf{c}_n)$ at control points on the interior of $\Omega$ by

\small{
\begin{equation}
 \label{eq:BC}
 f(\mathbf{x}) = \sum_{i=1}^{n}w_{i}(\mathbf{x})f(\mathbf{c}_{i})
\end{equation}
}
\\
here, except for the properties satisfied in many barycentric coordinate schemes, like reproduction, partition of unity and non-negativity, they prefer a target functional that reflects locality and smoothness for the coordinate functions while still convex.

For a function $w_{i}$ and a given value $s$, denote by $\{w_{i}>s\}:=\{\mathbf{x}|w_{i}(\mathbf{x})>s\}$ and $\{w_{i}=s\}:=\{\mathbf{x}|w_{i}(\mathbf{x})=s\}$ the $superlevel~set$ and the $level set$ of $s$, respectively.
Locality requires the area of the superlevel set $\{w_{i}>0\}$ to be small which means that the vector$[w_{1}(\mathbf{x}), \cdots, w_{n}(\mathbf{x})]$ is sparse, while for smoothness it is necessary that all curves/surfaces $w_{i}=\textrm{const}$ are smooth.

Based on these observations, the locality and smoothness of $w_{i}$ can be obtained using a functional that measures the sum of the perimeters of superlevel sets $\{w_{i}>s\}$ for all $s$.
And then the perimeter of each superlevel set regularizes the smoothness of its boundary level curve/surface, while the perimeter of $\{w_{i}>0\}$ penalizes the area of the influence region. It turns out that this functional is exactly the total variation of $\{w_{i}\}$. Finally, the problem is formulated as

\small{
\begin{equation}
 \label{eq:continuousLBC}
 \begin{aligned}
 & \min_{w_1, \cdots, w_2} \sum_{i=1}^{n} \int_{\Omega}^{} |\bigtriangledown w_{i}| \\
 & ~~~\mathrm{s.t.}~ \sum_{i=1}^{n}w_{i}(\mathbf{x})\mathbf{c}_{i}=\mathbf{x},
        \sum_{i=1}^{n}w_{i}=1,~w_{i}\geq0,~\forall \mathbf{x}\in\Omega,\\
 & ~~~~~~~~w_{i}(\mathbf{c}_{j})=\delta_{ij},~forall i,j,\\
 & ~~~~~~~~w_{i}~\textrm{is linear on cage edges and faces }\forall i.
 \end{aligned}
\end{equation}
}
\\
Discretely, after triangulating the domain $\Omega$, each $w_{i}$ is represented a function that is linear within each cell(triangle in 2D or tetrahedron in 3D) and then the gradient of $w_{i}$ is constant on each cell. Let $\mathcal{C}$ be the set of cells in the triangulation, the target functional~\ref{eq:continuousLBC} finally becomes

\small{
\begin{equation}
 \label{eq:discreteLBC}
 \sum_{T\in\mathcal{C}}^{}  \sum_{i=1}^{}
 \phi{_{i}^{T}}  A_{T}  \| \nabla T w_{i} \|_2,
\end{equation}
}
\\
where $\nabla T w_{i}$ is the gradient of $w_{i}$ in cell $T$, $A_{T}$ is the area(volume) of $T$, and $\phi{_{i}^{T}}$ is the value of the weighting function $\phi_{i}$ at the centroid of $T$.

Figure~\ref{fig:LBC} shows a cage-based deformation example with lower computational and storage requirement since each point on the target shape is only determined by a small number of control points.
Whatever, from the observation, we can see that there is a trade-off between locality and smoothness which is a common troubling issue for so many existing works.

\begin{figure}[ht]
  \centering
  \includegraphics[width=3in]{images/LBC_L1}
  \caption{Sparse regularization: TV based local barycentric coordinates\cite{zhang2014local}. Using LBC for 3D cage-based manipulation allows for local, smooth and shape-aware deformations. Only parts near the manipulated control points are deformed, as indicated by the color-coding.}
  \label{fig:LBC}
\end{figure}
  % shape matching

% considering important propertices, like their structural stability, aesthetics of a surface
%Especially in the case of modeling man-made structure like architecture or machine parts, geometric constraints are able to create and preserve ubiquitous alignment properties like element parallelism, conllinearity, fixed angles and distances, or symmetry relations.
%
%Generally, in an interactive 3D modeling system,
%it is crucial for each atomic editing operation to adjust as few auxiliary vertices as possible in order not to destroy the user's earlier editing effort,
%and the whole constraint resolution pipeline is required to run in real-time to enable a fluent, interactive workflow.
%
%To address both issues, \cite{habbecke2012linear} presents an interactive constrained modeling with a well-defined strategy that, for an atomic editing operation, computes $as~small~as~possible$ model updates in terms of the total number of adjusted vertices.
%
%For a model instance $\mathbf{X}^0$ whose elements are the vertex positions,
%a vector-valued functions $\mathbf{c}(\mathbf{X}^0)=\mathbf{0}$ denotes the satisfaction for all constraints.
%Then for a given editing displacement $\mathbf{d}$ corresponding to one user editing operation, the central goal is to find a correction displacement $\mathbf{d'}$  such that $\mathbf{c}(\mathbf{X}^0+\mathbf{d}+\mathbf{d'})=\mathbf{0}$, where the zero elements of $\mathbf{d'}$ and $\mathbf{d}$ are disjoint and $\mathbf{d'}$ should be as sparse as possible.
%
%Suppose the space of possible movements of each vertex $\mathbf{x}_{i}$ is represented with a basis $\{\mathbf{b}_{i,1},\mathbf{b}_{i,1},\mathbf{b}_{i,3}\}$, $\mathbf{b}_{i,k} \in \mathbb{R}^3$,
%they extend the 3-dimensional basis vectors to vectors $\mathbf{B}_{i,k}:=(0,...,0,\mathbf{b}{_{i,k}^{T}},0,...,0) \in \mathbb{R}^{3n}$ to represent the correction displacement $\mathbf{d'}$ with a linear combination
%
%\small{
%\begin{equation}
% \label{eq:ConstrainedModeling1}
% \mathbf{d'} := \sum_{i\notin I(\mathbf{b})}^{}\sum_{k=1}^{3}\alpha_{i,k}\mathbf{B}_{i,k}.
%\end{equation}
%}
%\\
%then the computation of the correction of displacement $\mathbf{d'}$ is actually to compute the non-zero coefficient $\alpha_{i,k}$.
%They firstly determine its non-zero set in analysis phase by solving
%
%\small{
%\begin{equation}
% \label{eq:ConstrainedModeling2}
% \sum_{i\notin I(\mathbf{d})}^{}\sum_{k=1}^{3}\alpha_{i,k}P\mathbf{B}_{i,k}=-P\mathbf{d}.
%\end{equation}
%}
%\\
%using $Orthogonal~Matching~Pursuit$(OMP)\cite{tropp2007signal},
%and here $P$ is the constraints' Jacobian $J_{\mathbf{c}}\in \mathbb{R}^{m\times 3n}$.
%
%Then the solution phase is designed to compute its values as
%
%\small{
%\begin{equation}
% \label{eq:ConstrainedModeling3}
% E(\{\alpha_{i,k}|(i,k)\in \Lambda\})=\sum_{j\in C}^{}c{_j^2}(\mathbf{X}_0+\mathbf{d}+\sum_{}^{}\alpha_{i,k}\mathbf{B}_{i,k}),
%\end{equation}
%}
%\\
%Figure~\ref{fig:constraint modeling} shows one modeling result. In this paper, each editing operation is performed on a input model instance that satisfies all predefined constraints, due to the sparsity of solutions, this strong assumption can not result in some limitations. But changing the predefined constraint types or the number of the constraints may result in failure cases.
%
%\begin{figure}[ht]
%  \centering
%  \includegraphics[width=3in]{images/modeling_L0}
%  \caption{Sparse regularization: constrained modeling\cite{habbecke2012linear}. Left: original configuration. Center: editing operation such that the base plane of the dormers changes its orientation(example A1). Right: the dormers' base plane does not change(example A2). Blue vertices are relaxed in the analysis phase and automatically updated by the editing system.}
%  \label{fig:constraint modeling}
%\end{figure}
%
