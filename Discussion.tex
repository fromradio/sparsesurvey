\section{Discussion}
\label{sec:Discussion}

Sparse representation shows its dominance in many geometry processing due to the robustness to noises and outliers as well as features preserving.
Actually, in summary, it can be classified into two main groups:

\paragraph{(1)}
From the geometry itself, finding the intrinsic sparse characteristic like curvature;
\paragraph{(2)}
If the intrinsic characteristic is not sparse, then using some differential operator, like Laplacian, to transform it into another characteristic space in which the transformed characteristic is sparse;

In this case, paying more attention to analyze some kind of geometric information carefully, such as Interaction Bisector Surface(IBS) and Beltrami number, may lead to a new exciting sparse problem.

However, due to the nature of sparsity modeling, the high computational cost is a common problem in most works with the iterative solving process.
Though they are designed for performance, they will be less meaningful if they cannot be applied to the practical applications.
So how to solve the sparse optimization problem while preserving the existing good properties is an worthy researching direction which will also maximize the values of many previous works.


%Many robust point cloud consolidation or surface reconstruction techniques have been developed to deal with a variety of acquisition errors like noise, outliers, missing data(holes) or registration artifacts.
%Most of the $\ell_1$ techniques are typically too expensive to achieve interactive reconstruction times for at least moderately sized point sets, even for parallel implementations,
%and are designed for quality rather than performance due to their nature. So the performance problem is still challenging and worth researching.
